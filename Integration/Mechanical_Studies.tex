\section{Mechanical structure studies}
%\writer{Felix Sefkow, Henri Videau, Karsten Buesser, Roman Poeschl, Toshiaki Tauchi}{3}
\label{ild:sec:mechanical_structures}

The mechanical behaviour of the ILD components is crucial in two respects. On the one hand, the high-precision and hermeticity of the detector requires a precise relative adjustment of subdetector components within each other, with tight tolerances at the interfaces and boundaries. These aspects were studied by simulating static deformations of the components under gravity and other constraints. On the other hand large devices in Japan must obey strict rules as regards their behaviour in case of seismic events (see section 6.9). This was investigated by modelling the dynamic behaviour of components, including the computation of their "eigen modes" and their reaction to reference earthquake parameters from the foreseen ILC Kitakami site. Most of the attention has up to now been given to the calorimeter mechanical structure which governs the global stiffness of the ILD detector inside the coil. %Some evaluations have also started for other subdetectors such as the TPC.

%\subsection{Calorimeter structure}

As mentioned in section 5.1.2 two options of the hadronic calorimeter, so-called "Videau" and "TESLA", are under consideration. In both cases, the electromagnetic modules are fixed to the inner plates of the hadronic wheels with two rails parallel to the z direction. Two critical aspects are of particular importance for the calorimeters: the respect of the tolerances of the thin azimutal clearance (2.5mm) between the electromagnetic modules, to avoid mutual contact and possible damage of the modules, and the flatness of the hadronic absorber plates which define the gaps in which the sensitive layers are introduced. The latter is particularly important for the SDHCAL instrumentation option since RPC's require a high level of flatness. Both Videau and TESLA mechanical options have been simulated in detail and the results provide input for further optimization of the layouts.

\subparagraph{\textbf{Videau simulations:}} The static behaviour of a full Videau calorimeter wheel was simulated with a shell model. In order to save computing time, the electromagnetic modules were approximated by a 3D solid model. This simplification was validated separately by a comparison to a single module shell model simulation. The results are shown in Figure~\ref{fig:integration:Videau_deformations}: the hadronic structure turns out to be very stiff with largest deformations of a fraction of a mm. This is due to the vertical flanges of the Videau modules which strongly rigidify the overall structure. The electromagnetic modules are also only slightly distorted: the largest deformation stays below 1mm and the azimutal clearance between modules is reduced to 2.3 mm in the worst case, well within the required tolerances. 

One advantage of the Videau layout is to avoid a projective dead zone at a polar angle of $90^0$, but the number of dead zones in z is 4 for the baseline number of 5 wheels. One question is whether one could reduce this number of dead zones by reducing the number of wheels to 3. Mechanical simulations show that for 3 wheels it is possible to keep deformations at the boundary with the electromagnetic modules within specifications, provided the inner hadronic absorber plate thickness is slightly increased.

First dynamic simulations of the Videau structure where also performed with the computation of eigen modes. They show that the overall calorimeter barrel behaves as a rigid structure under oscillating accelerations, with little variation of the z distance between the wheels. 


\begin{figure}[t!]
\centering
\includegraphics[width=1.0\hsize]{Integration/fig/Videau_deformations.jpg}
\caption{\label{fig:integration:Videau_deformations}Static deformations of the calorimeter "Videau" structure (right) and zoom on the electromagnetic modules displacements with a magnification factor of 750 (left).}
\end{figure}

\subparagraph{\textbf{TESLA simulations:}} 

In the ILD DBD the TESLA mechanical layout had corners at $0^{o}$, $45^{o}$, $90^{o}$ etc. in azimuth, which corresponded to the optimal configuration as regards mechanical stiffness. In order to facilitate the silicon sensor tiling of the ECAL endcap, the structure has since then been rotated by $22.5^{o}$. Since there are no vertical disks, the behaviour of the structure now resembles that of a Roman arc. 

The mechanical behaviour of the barrel structure with the new orientation has been simulated with shell and 3D models. The simulations showed deformations of up to 5~mm, requiring actions to reduce them. A minimised number of 10~mm wide additional spacers have been introduced between absorber plates in critical points -- near the fixtures of the ECAL and near the supports from the cryostat -- (Figure~\ref{fig:integration:Tesla_deformations} left), resulting in static deformations of less than 2~mm everywhere (Figure~\ref{fig:integration:Tesla_deformations} right). 

\begin{figure}[t!]
\centering
\includegraphics[width=1.0\hsize]{Integration/fig/Tesla_studies.jpg}
\caption{\label{fig:integration:Tesla_deformations}Zoom on the consolidation of the TESLA structure with spacers (left) and resulting static deformations of the structure (right).}
\end{figure}

The 3D model has also been used to obtain the first 200 vibration eigen modes. However, calculating the response of the system to an external excitation exceeds computational limitations and requires more advanced methods. 
The so-called component mode synthesis (CMS) method has been used, in which the detailed sub-structure of calorimeter modules is replaced by blocks which however preserve the overall mechanical behaviour of the module at specially configured boundaries. 
A simplified AHCAL model has been used to validate this model against a full 3D simulation up to the response to frequency sweeps and to real Japanese Earth quake data. 
The application of this model to the re-optimised AHCAL structure is in progress. 

\subparagraph{\textbf{Simulations cross checks:}} 

The detailed simulation models of the Videau and TESLA structures have been shared between LLR and DESY in order to allow mutual cross-checks of the results in the future. 

%\subsection{Other subdetectors}

%\textit{If available, results on the mechanical behaviour of other subdetectors (e.g. TPC) are also welcome.}

\vspace{2cm}
