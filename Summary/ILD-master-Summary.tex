\chapter{Summary}

In this paper the main design ideas and the state of the implementation of the design for the ILD detector have been discussed. The information in this paper is an update of the previous ILD documents, in particular the DBD. 

The ILD concept presented in this paper is the result of an intense effort of many people over a significant length of time. The detector proposed is developed to a point that we are convinced that there are no insurmountable problems building ILD. Even though a significant amount if engineering remains to be done, the concepts for a realisation of ILD have been developed, and are demonstrated in this paper. 

ILD has always worked in very close cooperation with major R\&D collaborations like CALICE, LCTPC or FCAL. The solutions proposed for ILD have all been reached by a common effort between the R\& D collaborations the ILD. In nearly all cases significant demonstrations and proof-of-concept experiments could be done, and are used to validate the soundness of the designs of ILD. 

The ILD concept relies on particle flow as the central method to reconstruct events at an ILC. This has far-reaching consequences for the design of the detector. An imaging calorimeter, with a large number of cells both in transverse and in the longitudinal direction, is essential. This very powerful calorimeter is combined with a tracking system which combines powerful and highly redundant tracking with excellent vertexing capabilities. The complete tracking system has been optimized for low material, for an as nearly as possible $4 \pi$ acceptance.

Together with the R\&D collaborations the past few years since the publication of the DBD have been used to further develop the technologies, but more importantly, push the different technologies closer to a point where their readiness for use in a large scale detector have been shown. This is been particularly successfully done for the very ambitious ILD calorimeters, where the different technologies proposed all have recently tested large scale prototypes, build in a way which would be scalable to the complete experiment. 

ILD has developed two versions of its detector concept, a large and a small version. In this document, many results are shown as a function of these two models, as a means to understand better the scaling of both performance and cost of ILD. No decision has been taken at this moment as to how the final ILD detector would look like, whether it would follow more closely the large or the small design. This will also depend largely on the further development of the ILC project, its ultimate definition, and the boundary conditions set to realise the project. 

Both versions of ILD have been the subject of study for a number of bench-marking reactions. These were selected in a way to probe the capabilities of the two designs as precisely as possible. The impact the detector design has on the results is clearly demonstrated in the bench-marking section of this document. 

In addition ILD has pursued a range of analyses to probe and demonstrate the potential of ILC (and ILD). These analyses are not all described in this document, but are available from the recent review on science at the ILC \cite{ild:bib:ref-keisuke}.

In its current form the ILD detector concept is a well proven and tested concept for a detector at an electron-positron collider, and in particular at the ILC collider. It is based on detailed simulations, and on an comprehensive body or R\&D work on sub-detector technologies. All proposed technologies are demonstrated and have shown that they can be used in this experiment. ILD thus is ready to go forward to proposing a detector at an electron-positron collider in general, ILC in particular, once the facility is moving into the next project phase. 






