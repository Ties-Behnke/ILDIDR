\chapter{Summary and outlook}
\label{chap:summary}

This paper presents the concept and the design status of the ILD detector. Its content provides a self-consistent update of the previous ILD documents, in particular the DBD~\cite{ild:bib:ilddbd}. 

The updated ILD concept is the result of an effort of many people over a significant time. The proposed detector is now developed to a point where all construction issues are under control, even though a significant amount of engineering remains to be done. 
%the concepts for a realisation of ILD have been developed, and are demonstrated in this paper. 

ILD has always worked in very close cooperation with major R\&D collaborations such as CALICE, LCTPC or FCAL. Many solutions proposed for ILD result from this common effort. In most cases significant demonstrations and proof-of-concept experiments could be done, and are used to validate the soundness of the designs of ILD. 

The ILD concept relies on particle flow as the central method to reconstruct events at an ILC. This has far-reaching consequences for the design of the detector. An imaging calorimeter, with a large number of cells both in transverse and in the longitudinal direction, is essential. This very powerful calorimeter is associated to a tracking system which combines powerful and highly redundant tracking with excellent vertexing capabilities. A central and unique part of the ILD concept is a large volume time projection chamber in the tracking system, which delivers excellent efficiency for the tracking over a large solid angle. The complete tracking system has been optimised for low material and a solid angle acceptance as close as possible to $4 \pi$.

Together with the R\&D collaborations, the past few years since the publication of the DBD have been used to further develop the subdetector technologies and, more importantly, push the different technologies closer to a point of readiness for use in a large scale detector. This has been particularly successful for the very ambitious ILD calorimeters, where most proposed technologies have recently tested large scale prototypes, built in a way which would be scalable to the complete experiment. 

ILD has developed two versions of its detector concept, a large and a small version. In this document, many results are shown as a function of these two models, as a means to better understand the scaling of both performance and cost of ILD. No decision has been taken at this moment as to whether the ILD detector would follow more closely the large or the small design. This will also depend largely on the further development of the ILC project, its ultimate definition, and the boundary conditions set to realise the project. 

Both versions of ILD have been the subject of study for a number of bench-marking reactions with detailed simulations. These were selected in a way to probe the capabilities of the two designs as precisely as possible. The impact of the detector design on the results is clearly demonstrated in the bench-marking section of this document. 

In addition ILD has pursued a range of analyses to probe and demonstrate the physics potential of ILC (and ILD). These analyses are not all described in this document, but are available from the recent review on science at the ILC \cite{ild:bib:ref-keisuke}.

In its current form the ILD detector concept is a well proven and tested concept for a detector at an electron-positron collider, and in particular at the ILC collider. 
%It is based on detailed simulations, and on a comprehensive body of R\&D work on subdetector technologies. 
%All proposed technologies are demonstrated and have shown that they can be used in this experiment. 
In the coming years the ILD subdetector technologies will be further consolidated thanks to the ongoing construction of new detectors, out of which several spin-offs of ILC-oriented R\&D. Among current projects of particular interest for ILD are the ALICE upgrade, CBM and BELLE-II vertex and inner tracker detectors (Si-pixel technology), the ALICE upgrade and T2K/ND280 TPCs (TPC readout technologies) and the CMS upgrade HGCAL detector (calorimetry technologies). The ILD Collaboration is also currently performing a full simulation of the ILC collisions at a c.m.s energy of 250 GeV (as opposed to the 500 GeV collisions used in the present benchmarking), in order to revisit the detector performance in its latest design for the baseline ILC program. This will also allow to assess the potential impact of possible new features such as extended pixel tracking, a TOF functionality or a calorimetry high-resolution timing.  
ILD therefore remains committed to go forward to proposing a detector at an electron-positron collider in general, ILC in particular, once the facility is moving into the next project phase. 

At the time of writing this report, no final decision on the construction of the ILC accelerator or any other electron-positron collider has been reached. Japan has announced that it will study the ILC proposal in depth, and has started discussions with international partners to understand possible project schemes, including cost sharing issues. At the same time, in Europe, the ongoing discussion on the particle physics strategy for the next decade, which includes several options for electron-positron colliders at the energy frontier, is coming close to conclusion. 
%A result which might include also recommendations on future colliders to be built, is expected for 2020. 
This paper therefore provides a timely documentation of the current state of the ILD detector, and defines a possible configuration with which ILD could enter into a proposal at the ILC, should ILC move forward. 




