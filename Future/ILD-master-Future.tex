\chapter{Future of ILD}

The ILD detector concept has been developed for electron-positron collisions at energies between 90~\GeV and about 1~\TeV. Most elements of ILD apply to all discussed electron positron facilities. The CLIC detector concept and the concepts discussed at CEPC have many commonalities with ILD. A particular effort has been made to optimise ILD for use at the ILC, including adjustments to the particular properties of the collision time structure and beam parameters. 

At the time of writing this report, no final decision on the construction of the ILC accelerator has been reached. Japan has announced that it will study in depth the ILC proposal, and will start negotiations with international partners to understand possible project schemes, including cost sharing issues. At the same time, in Europe, the discussion on the particle physics strategy for the next decade has started. A result which might include also recommendations on future colliders to be built, is expected for 2020. This document thus is a timely work to document the current state of the ILD detector, and define a possible configuration with which ILD could enter into a proposal at the ILC, should ILC move forward. 

The future of the ILD group will obviously depend on the results of these international discussions, and decisions. 