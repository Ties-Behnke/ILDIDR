\subsection{Time projection chamber}
\writer{Paul Colas, Akira Sugiyama}{3}

The ILD TPC R\&D is being conducted mainly within the LCTPC Collaboration [ref]. 

The workhorse for validation of detector prototypes and operational conditions is the TPC test set-up installed permanently in the DESY test beam (Figure~\ref{fig:det:TPC_test_setup}). The TPC stands within a magnet providing a magnetic field of 1 Tesla, and the beamline is equiped with precise incident and outgoing particle beam telescopes allowing to quantify the TPC reconstruction precision as function of the particle parameters. The beamtest set up is currently being upgraded with the high precision LYCORIS silicon telescope [ref], and a new TPC field cage with reduced field distorsion is being assembled.

\begin{figure}[t!]
\centering
\includegraphics[width=1.0\hsize]{Detector/fig/TPC_test_setup.jpg}
\caption{The TPC test setup at DESY. The insert shows the geometrical structure of the TPC endcap which can host prototypes of detection planes.}
\label{fig:det:TPC_test_setup}
\end{figure}

Significant progress has been performed in the manufactoring process of detection modules for each of the readout options. A new micromegas layout with resistive anodes has been shown to exhibit reduced boundary distorsions [ref]. The flatness of the GEM modules has been improved significantly, increasing the gain uniformity by a factor 2 [ref]. Operational GridPix "QUAD" modules have been built based on the TimePix3 pixel chip [ref]. Recent prototypes of the three types of detection modules are shown in Figure~\ref{fig:det:TPC_prototypes}.  

\begin{figure}[t!]
\centering
\includegraphics[width=1.0\hsize]{Detector/fig/TPC_prototypes.jpg}
\caption{TPC prototype detection modules for the three baseline technologies under consideration: micromegas module (left), GEM module (middle) and GridPix QUAD module (right).}
\label{fig:det:TPC_prototypes}
\end{figure}

The performance of the three technologies has been measured in beam tests. Figure~\ref{fig:det:TPC_performance} shows that the target goals of a spatial resolution better than 100 microns and of a dE/dx resolution better than 5\% have been reached in all cases.   

\begin{figure}[t!]
\centering
\includegraphics[width=1.0\hsize]{Detector/fig/TPC_performance.jpg}
\caption{Resolution on the track position (left) and ionisation loss dE/dx (right) as function of the drift length, for the three readout options under consideration (\textit{figures to be updated with inclusion of the 3 technologies})}
\label{fig:det:TPC_performance}
\end{figure}

\vspace{1cm}
Two critical aspects of a TPC operation consist in the cooling of the readout endcaps, which must be realized with minimal dead material, and the mitigation of the drift field distorsions which may develop from the accumulation of ions in the drift volume. For the first point a double phase $CO_2$ cooling system with thin low-material fluid pipes has been developed and shown to perform adequately (Figure~\ref{fig:det:TPC_performance} left). For the second point an ion gating scheme based on GEM foils has been implemented and beam tested. Results show that a good transparency for drift electron signals can be maintained while preventing the accumulation of ions in the drift volume (Figure~\ref{fig:det:TPC_performance} right).   

\begin{figure}[t!]
\centering
\includegraphics[width=1.0\hsize]{Detector/fig/TPC_operation.jpg}
\caption{TPC operation achievements: temperature stability with double phase $CO_2$ cooling (left) and signal electron transparency with GEM gating (right).} 
\label{fig:det:TPC_operation}
\end{figure}

\vspace{2cm}