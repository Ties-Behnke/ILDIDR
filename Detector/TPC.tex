\subsection{Time projection chamber}
%\writer{Paul Colas, Akira Sugiyama}{3}

The ILD TPC R\&D is being conducted mainly within the LCTPC Collaboration \cite{ild:bib:TPC_lctpc}. 

The workhorse for validation of detector prototypes and operational conditions is the TPC test set-up installed permanently in the DESY test beam\cite{ild:bib:TPC_desytb} (Figure~\ref{fig:det:TPC_test_setup}). The TPC is situated in a magnet providing a magnetic field of 1 Tesla, and the beam line is equipped with precise incident and outgoing particle beam telescopes allowing to quantify the TPC reconstruction precision as function of the particle parameters. The beam test set up is currently being upgraded with the high precision LYCORIS silicon telescope~\cite{ild:bib:TPC_lycoris}, and a new TPC field cage with reduced field distortion is being assembled.

\begin{figure}[t!]
\centering
\includegraphics[width=1.0\hsize]{Detector/fig/TPC_test_setup.jpg}
\caption{The TPC test setup at DESY. The insert shows the geometrical structure of the TPC end cap which can host prototypes of detection planes.}
\label{fig:det:TPC_test_setup}
\end{figure}

Significant progress has been seen in the manufacturing process of detection modules for each of the readout options. A new micromegas layout with resistive anodes has been shown to exhibit reduced boundary distortions\cite{ild:bib:TPC_distortions}. The flatness of the GEM modules has been improved significantly, increasing the gain uniformity by a factor 2 \cite{ild:bib:TPC_GEMflatness}. Operational GridPix "QUAD" modules have been built based on the TimePix3 pixel chip\cite{ild:bib:TPC_quad}. Recent prototypes of the three types of detection modules are shown in Figure~\ref{fig:det:TPC_prototypes}.  

\begin{figure}[t!]
\centering
\includegraphics[width=1.0\hsize]{Detector/fig/TPC_prototypes.jpg}
\caption{TPC prototype detection modules for the three baseline technologies under consideration: micromegas module (left), GEM module (middle) and GridPix QUAD module (right).}
\label{fig:det:TPC_prototypes}
\end{figure}

The performance of the three technologies has been measured in beam tests. Figure~\ref{fig:det:TPC_performances} shows the measured point resolution in 1 T magnetic field for drift distances from 0 to 0.6 m. This can be safely extrapolated to $\sim~100$ $\mu$m in a field of 3.5~T at a drift length of 2.3~m. The dE/dx resolution by the truncated mean method has been measured to be respectively $4.6\%, 4.5\%$ and $4.2\%$ for Micromegas, GEM and GridPix technologies. It improves to $3.8\%$ for GridPix using a cluster counting method. The conclusion is that the target requirement of a spatial resolution of 100 microns and an aim of a dE/dx resolution better than 5\% have been reached in all cases.   


\begin{figure}[t!]
\centering
\includegraphics[width=1.0\hsize]{Detector/fig/TPC_performances.jpg}
\caption{Resolution on the track position in r$\phi$ (left) as function of the drift length and resolution on the ionisation loss dE/dx (right) as function of the track length, for the three readout options under consideration.}
\label{fig:det:TPC_performances}
\end{figure}

\vspace{1cm}

Two-track separation has also been investigated. A $47\% \rm{X^0}$ steel target was introduced near the TPC wall to produce multi-track events suitable for this study.
From these events a 2-hit separation distance of 4 to 6 mm was observed depending on the drift distance. An algorithm based on fitting the double-hit charge deposition with expected Pad Response Function width allowed this separation distance to be reduced to 2mm, with 1.3 mm pads (Figure~\ref{fig:det:TPC_separation}).

\begin{figure}[t!]
\centering
\includegraphics[width=0.7\hsize]{Detector/fig/TPC_separation.jpg}
\caption{2-hit separation as a function of the drift distance from beam test. Red dots are the results for standard hit reconstruction. Blue triangles show the results of the improved algorithm}
\label{fig:det:TPC_separation}
\end{figure}


\vspace{1cm}
A critical aspect of a TPC operation consists in the cooling of the readout end caps, which must be realised with minimal dead material.  For this a double phase $\rm{CO}_2$ cooling system with thin low-material fluid pipes has been developed and shown to perform adequately. 

Also critical is the mitigation of the drift field distortions which may develop from the accumulation of ions in the drift volume. For this an ion gating scheme based on a large aperture GEM foil (Figure~\ref{fig:det:TPC_gating} left) has been implemented and beam tested\cite{ild:bib:TPC_gatinginbeam}.
Results show that a good transparency for drift electron signals can be maintained while preventing the accumulation of ions in the drift volume\cite{ild:bib:TPC_gatingpaper} (Figure~\ref{fig:det:TPC_gating} right).

\begin{figure}[t!]
\centering
\includegraphics[width=1.0\hsize]{Detector/fig/TPC_gating.jpg}
\caption{TPC gating: picture of a GEM gating grid (left) and signal electron transparency with GEM gating (right).} 
\label{fig:det:TPC_gating}
\end{figure}

In conclusion all basic aspects of the operation have been shown to meet the requirements for the ILC. 
\vspace{3cm}