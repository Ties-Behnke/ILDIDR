\chapter{Detector Layout and Technologies}

\section{Overall structure of the detector}
\writer{Claude Vallee, Karsten Buesser}{1}
\vspace{4cm}
\subsection{Global structure and parameters}
\writer{Claude Vallee, Karsten Buesser}{1}

Reminder of the global structure of the ILD detector, focusing details on the main changes since DBD, and mentioning remaining open options like anti-DID.

\vspace{2cm}
\subsection{Subdetecor layouts}
\writer{Subdetector technical conveners}{4}

Description of the latest baseline design of subdetectors, including open options. Each technology option description should indicate its advantages (pro) and critical aspects (cons) in respect to ILD specifications. Potential new capabilities for the future (e.g. calorimeter timing) should also be indicated.

Vertex detector (Besson, Ishikawa, Vos): nb layers, dimensions, technology options.

Internal silicon trackers (Besson, Vos, Vila): FTD disks, dimensions, technology per disk, SiT including pixel option, short discussion status
outer TPC  silicon layers

TPC (Colas, Sugiyama): structure and readout options (GEM, Micromegas)

ECAL (Brient, Ootani): mechanical layout and readout options (Si, Sc)

HCAL (Laktineh, Sefkow): mechanical options (Videau,TESLA and readout options (Sc, RPC)

VFS (Benhammou, Schuwalow): layout adapted to new L*, LUMICAL, BEAMCAL,LHCAL sensors

Iron instrumentation (Saveliev): sensitive layers in yoke baseline design and sensor structure

\vspace{2cm}
\section{Subdetector technology status}
\writer{Subdetector conveners}{}

This section is one of the main technical added values of the IDR. It should summarize all technological progress since the DBD, including beam tests of technological prototypes and ongoing spinoffs. It should also indicate the remaining steps to fulfil the ILD requirements, with a focus of critical aspects associated to each technology choice. For conciseness only the highlights should be illustrated, with all details to be referenced in technical publications. As regards illustrations it is proposed to stick to o(1-2) photo and o(1-2) plot  for each technology. 

\vspace{2cm}
\subsection{Vertex detector}
\writer{Auguste Besson, Akimasa Ishikawa, Marcel Vos}{3}

CMOS spin-offs: ALPIDE (ALICE upgrade) and MIMOSIS (CBM), New PSIRA chip for ILD

DEPFET spin-off: BELLE-2, BEAST and first physics events

FPCCD long prototype and irradiation tests

Short mention of other options:  SOI, etc

Low material support developments (PLUME ladder) and cooling studies

\vspace{2cm}
\subsection{Silicon inner tracking detectors}
\writer{Marcel Vos, Ivan Vila}{1}

Above DEPFET developments and FTD thermo-mechanical mockup

\vspace{2cm}
\subsection{Time projection chamber}
\writer{Paul Colas, Akira Sugiyama}{3}

TPC prototype for generic beam tests of all readout options, the gating scheme and cooling. Mention LYCORIS silicon telescope and new field cage in construction.

CO2 Cooling measurements            

Successful gating achieved with GEM

GEM and Micromegas spacial and ionization resolutions

Ongoing technological developments: improved module planarity, GRIDPIX RO options, etc.  

\vspace{2cm}
\subsection{Calorimeters}
\writer{Jean Claude Brient, Wataru Ootani, Felix Sefkow, Imad Laktineh}{5}

Si-ECAL: technological prototype (incl long slab) beamtest results, CMS HGCAL spinoff.

Sc-ECAL results from new detector unit in construction.

AHCAL beamtest results from large technological prototype, CMS HGCAL spinoff.

SDHCAL beamtest results of large technological prototype at CERN, ongoing development of large chambers. 

\vspace{2cm}
\subsection{Very forward detectors}
\writer{Yan Benhammou, Sergej Schuwalow}{2}

LUMICAL sensors: Beam test of thinner prototypes with improved transverse resolution of e-showers

BEAMCAL and LHCAL developments

\vspace{4cm}
\subsection{Iron instrumentation}
\writer{Valery Saveliev}{1}

Fermilab Scintillator detectors prototypes and performance tests

\vspace{2cm}
