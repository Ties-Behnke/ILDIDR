\chapter{ILD as an Organisation}
The ILD detector concept emerged from the merger of two earlier detector concept, the GLD (Global Large Detector) and the LDC (Large detector concept) groups. GLD had its basis mostly in Asia, LDC in Europe. In 2008 both groups decided to join forces to propose a detector at the at this time new ILC collider proposal. 

Over the years ILD developed into a proto-collaboration with an internal structure and some simple rules, to organised the common work. After the delivery of the DBD, and facing the development of ILC towards a project in Japan, ILD gave itself a new set of bylaws and a more well defined structure, which closely resembles that of established collider experiment collaborations. 

The central body of the collaboration is the institute assembly (IA), in which each participating institute is represented by one vote. The IA decides on all matters of fundamental importance for ILD, decides on ILD membership, and elects or confirms all ILD positions. 

ILD is represented by a spokesperson and a deputy. Both are elected from the institute assembly. Coordinators take care of the different areas of activities in ILD, with currently a technical coordinator, a physics coordinator, and a software coordinator. Together with 3 elected members these coordinators and the spokespeople form the Executive Team (ET), which is responsible for the day-to-day running of the group. 

