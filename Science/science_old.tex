
Our goal is to achieve a unified understanding of nature, which is to go back in time to the moment of creation, when everything, matter, force, and space-time, is thought to be unified. Astronomers study the early Universe by observing distant space with telescopes. This method, however, cannot penetrate into the very early universe before the recombination, i.e. 380 thousand years after the Big Bang. Energy frontier collider experiments provide a unique opportunity to investigate the Universe before the recombination by reproducing in a controlled manner reactions that could happen only in the early Universe. Our current understanding of the Universe is summarized in the Standard Model (SM) of particle physics. The SM consists of matter fermions (quarks and leptons), force carrying vector bosons (gauge bosons), and a scalar boson (Higgs boson) designed specifically to give masses, where needed, to otherwise massless SM particles by breaking the electroweak symmetry through Higgs condensation in the vacuum. The 2012 discovery of the Higgs boson at the LHC has completed the SM particle spectrum. 

Up to now, the SM has survived all the intense scrutiny through searches for new particles at energy frontier colliders such as LEP, Tevatron, and, most recently, the LHC, through precision measurements of the $W$ and $Z$ bosons, and through searches for anomalies in flavor physics. While being extremely successful, however, the SM has left many open questions such as what is the nature of dark matter and dark energy,  why matter, not antimatter, dominates the Universe, what is the origin of neutrino masses and mixings, and why did the Higgs field fill the entire Universe and why at the electroweak scale. These questions call for physics beyond the Standard Model (BSM). The Higgs boson discovery hence marked the start of a new voyage to the moment of creation.

Since the Higgs boson discovery, the LHC has so far seen no new particle or phenomenon, indicating that existing techniques to look for BSM physics at existing particle physics facilities are approaching their limits. This makes precision Higgs study of utmost importance. The ILC is going to fully exploit the Higgs boson as a new discovery tool and will explore the energies at and beyond the Electroweak (EW) scale, in order to directly address the question: why did the Higgs field fill the entire Universe and why at the EW scale, i.e. about one trillionth of a second after Big Bang. 
The key is precision measurements of Higgs couplings to various SM particles. In the SM the Higgs boson's couplings to various SM particles are proportional to their masses. BSM physics modifies this proportionality, leaving its nature (existence of another dimension, or a deeper stratum of matter, or multi-verse?) imprinted in the deviation pattern from the SM. 
%There are three possibilities. The first is the road of another dimension, supersymmetry (SUSY) or extra-dimension, where the Higgs boson is elementary. This road leads us directly to the ultimate unification. The second is the road of deeper stratum of matter, where the Higgs boson is composite, implying the existence of new strong force and many composite particles in the multi-TeV region. The third is the road where the SM, as it is, holds up to a very high scale. This road requires a new principle, such as the existence of multiverse. 
Since, in general, expected deviations are at most 10\%, to see the deviations and their pattern and to decide the future direction of particle physics, we need a percent level precision for various Higgs couplings.

The goal of the ILD experiment is to carry out Higgs coupling measurements, as well as measurements of  $W$ boson properties and fermion pair production with unprecedented precisions, while searching directly for new particles with unprecedented sensitivity.
To achieve this goal, the ILD detector has been designed to reconstruct every event at the level of quarks, leptons, and fundamental bosons including gauge bosons and the Higgs boson, so as to see events as viewing Feynman diagrams. For this purpose, the ILD design has been optimized for Particle Flow Analysis (PFA).
ILD has been designed to be an experiment at a collider providing electron-positron collisions between 90 GeV and 1 TeV center-of-mass energy. 
%The science goals of the ILC have been described in detail in \cite{ILCESU1}.
Electron positron collisions will allow us to take a new and different look at the physics of the Standard Model, and in particular, at probing the boundaries of the validity of the standard model. 
%Up to now, we have sought evidence for new interactions or new particles from direct searches for new particles at LEP, the Tevatron, and, most recently, the LHC, from measurements of the $W$ and $Z$ bosons, and from searches for anomalies in flavor physics.  We are now approaching the limits of these techniques with current particle physics facilities.  The ILC will extend our search capabilities in precision measurements of $W$ boson couplings and fermion pair production, and will provide new opportunities for the direct discovery of new particles.  But, most of all, it will open a completely new road through the  high-precision study of the Higgs boson. 
Indeed, the core physics program at the ILC is to make high-precision
 measurements of the properties of the Higgs boson, for a range of different center-of-mass energies. 
% The Higgs-boson is playing a central role in the Standard Model. Since its discovery at the LHC, a first determination of its properties has taken place, but many open questions remain. 
Many of the key properties of the Higgs are currently known only to the 10\% level. With the ILC, it will be possible to improve most of these errors to the 1\% level or below, to a point where the quantities become sensitive to radiative corrections. %In this way, knowing precisely the properties of the Higgs not only tells us about the least well-known particle in the Standard Model, but also provides an entry point into physics and science beyond the Higgs and beyond the Standard Model. 

%This approach has a great potential to contribute to the study and understanding of natures in areas where the standard model clearly is not enough.  It cannot for example address basic facts about the universe in the large, in particular, the excess of matter over antimatter and the origin of the cosmic dark matter. To make progress, we need observational evidence from particle physics of violations of the SM.  These will provide clues that can show the way forward.

In the context of the ILD group a broad range of studies have been undertaken to understand the potential of an experiment at the ILC, and of ILD as a concrete detector in particular. A very brief summary of the main results will be given in the remainder of this chapter. It is important to point out that 
 the analyses which are discussed here are based on fully simulated events, using a realistic detector model and advanced reconstruction software, and in many cases includes estimates of key systematic effects. This is particularly important when estimating the reach the ILD detector at the ILC will have for specific measurements. Determining, for example, the branching ratios of the Higgs boson at the percent level depends critically on the detector performance, and thus on the quality of the event simulation and reconstruction. 

In many cases the performance used in the physics analyses has been tested against prototype experiments. The key performance numbers for the vertexing, tracking and calorimeter systems are all based on results from test beam experiments. The particle flow performance, a key aspect of the ILD physics reach, could in the absence of a large scale demonstration experiment not be fully verified, but key aspects have been shown in experiments. This includes the single particle resolution for neutral and charged particles, the particle separation in jets, the linking power between tracking and calorimetry, and key aspects of detailed shower analyses important for PFA. 

While the physics case studies are mostyl based on the version of the ILD detector presented in the detector volume of the ILC DBD~\cite{ild:bib:ILDDBD}, ILD has recently initiated a systematic bench-marking effort to study the performance of the ILD concept, and to determine in particular the correlations between science objectives and detector performance. 
%The list of benchmark analyses which are under study is given in table \ref{tab-benchmark}. 
Even if the ILC will start operation at a center-of-mass energy of 250\,GeV, the ILD detector is being designed to meet the more challenging requirements of higher center-of-mass energies, since major parts of the detector, e.g.\ the coil, the yoke and the main calorimeters will not be replaced when upgrading the accelerator. Therefore, most of the detector benchmark analyses are performed at a center-of-mass energy of 500\,~GeV, and one benchmark even at 1\,TeV. The assumed integrated luminosities and beam polarisation settings follow the canonical running scenario~\cite{Barklow:2015tja}. 
In addition to the well-established performance aspects of the ILD detector, the potential of new features not yet incorporated in the existing detector prototypes, e.g.\ time-of-flight information, is being evaluated. 
