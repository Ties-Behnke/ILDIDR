\chapter{Introduction}
\label{chap:introduction}
%\writer{Ties Behnke, Kiyotomo Kawagoe}{2}
The ILD detector is proposed for an electron-positron collider with collision c.m.s energies from 90~\GeV~to about 1~\TeV. It has been developed over the last 10 years by an international team of scientists with the goal to design and eventually propose a fully integrated detector, primarily for the international linear collider, ILC.

The fundamental ideas and concepts behind the ILD detector have been discussed in two previous documents, the letter of intent \cite{ild:bib:ILDloi} and the detailed baseline document, DBD \cite{ild:bib:ilddbd}. 
The ILD concept has been scrutinized by international groups at different occasions. After the publication of the letter of intent \cite{ild:bib:ILDloi} an international expert team reviewed ILD alongside two other detector concepts, the SiD \cite{bib:sid:loi} and the fourth detector concept \cite{bib:4th:loi}. SiD and ILD were both validated as potential candidate experiments at the ILC. 

ILD has been designed as a multi-purpose detector. It should deliver excellent physics performance for collision energies between 90~\GeV~and 1~TeV. The detector has been optimized to perform excellently at the initial  energy of 250~\GeV, while maintaining full physics capabilities at higher energies. 

 A central element of the design is the capability of the detector to reconstruct precisely complex hadronic final states as well as events with leptons or missing energy in the final state. To achieve these goals, precision detector elements such as vertex detectors are combined with a large volume time projection chamber for excellent tracking efficiency and with a highly granular calorimeter, in an overall design philosophy called particle flow, developed for optimal global event reconstruction. 

In this document the current state of the design of the ILD detector is summarised. The technologies which are proposed for the different parts of the detector are introduced. An extensive benchmarking has been performed to demonstrate the physics performance of ILD. In order to ensure a detector adequacy for the whole ILC program, including possible future energy upgrades, benchmarking has been mostly done at a collision c.m.s energy of 500 GeV instead of the initial 250 GeV energy. Two detector configurations, a large and a smaller one, have been simulated in detail to provide guidelines towards an optimal balance between performance and cost. 
%Both concepts are the result of an optimization effort, with several goals - cost effectiveness was foremost a criterion for the smaller one, ultimate performance for the larger model. 

A lot of the work presented in this report is based on detector R\&D work which has taken place over the past decade to develop the necessary technologies. 
This work has been typically conducted within dedicated R\&D collaborations, which are independent but maintain very close connections to ILD. All technologies selected by ILD for its subsystems have been proven experimentally to meet the performance goals, or to come very close. Developing a very powerful detector concept over a long period of time requires balancing cutting edge technologies, which might become available while the concept is being developed, with safe and sound solutions. ILD in many cases is pursuing more than one technological option, to remain flexible and to be able to adapt to new developments. The concept group  wants to remain open and flexible to be prepared to select the most modern and most powerful technology once it is necessary. 
%However a distinction is made between options and alternatives: while options have undergone an extensive R\&D program and have passed critical proof-of-concept tests, alternatives are potentially interesting and promising technologies which have not matured to a similar level at the time of writing this document. 

The ILD concept group has currently 62 member institutes from all around the world. The group has evolved into a proto-collaboration, which is positioning itself to move forward with a proposal for a detector at the ILC or other proposed electron-positron facilities.

The document starts with a short review of the science goals of the ILC, and how the goals can be achieved today with the detector technologies at hand. After a discussion of the ILC and the environment in which the experiment will take place, the detector is described in more detail, including the status of the development of the technologies foreseen for each subdetector. The integration of the different sub-systems into an integrated detector is discussed, as is the interface between the detector and the collider. This is followed by a concise summary of the benchmarking which has been performed in order to find an optimal balance between performance and cost. To the end the costing methodology used by ILD is presented, and an updated cost estimate for the detector is presented. The report closes with a summary of the current status and of planned future actions. 

